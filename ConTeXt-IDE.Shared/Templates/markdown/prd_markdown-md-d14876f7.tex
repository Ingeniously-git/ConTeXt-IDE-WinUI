\section[title={Install Pandoc},reference={install-pandoc}]

\startitemize[packed]
\item
  Go to https://github.com/jgm/pandoc/releases/latest to install Pandoc.
\item
  Restart the IDE
\item
  In the IDE, Navigate to ConTeXt --> Manage Modules
\item
  Install the filter module
\stopitemize

\section[title={Compile Markdown -> TeX ->
PDF},reference={compile-markdown---tex---pdf}]

You can now write your texts in Markdown using the internal Markdown
previewer!

The content of the .md file gets parsed by Pandoc to a .tex file, which
gets included to your document and compiled by \ConTeXt.

You can use many \ConTeXt~commands \cite{hagen}, Pandoc will just pass
them through:

\startformula  y(x) = m \cdot x + c  \stopformula

\subsection[title={Text},reference={text}]

\startblockquote
Markdown font {\em attributes} will {\em show} in the {\bf preview} and
in the {\bf {\em compiled}} \overstrikes{pdf}.
\stopblockquote

Superscripts and subscript don't seem to work with Pandoc:

H3O^+

\subsection[title={Itemize},reference={itemize}]

The markdown live previewer seems to have some issues displaying nested
items:

\startitemize[packed]
\item
  Item 1
\item
  Item 2
  \startitemize[packed]
  \item
    Bullet 1 in list item 2
  \item
    Bullet 2 in list item 2
  \stopitemize
\item
  Item 3
\stopitemize

\subsection[title={Images},reference={images}]

In order to see Images in the preview and in the compiled output, please
use a path relative to the current .md file like so:

\placefigure{This is the figure
caption}{\externalfigure[pictures/NyquistPlot1.png][width=0.5\textwidth]}

Some stuff like link attributes are Pandoc-specific, that is why
\quotation{\{width=50\letterpercent{}\}} has no effect in the live
preview.

\subsection[title={Code},reference={code}]

\starttyping
public static void Main(string[] args)
{
  Console.WriteLine("Hello world!");
}
\stoptyping

The compile time (as always) scales with the document size, so you might
want to split your .md files (= individual sections) when they exceed
\lettertilde{}1000 lines.

\subsection[title={Subsections},reference={subsections}]

\subsubsection[title={Subsubsection},reference={subsubsection}]

\subsubsubsection[title={Subsubsubsection},reference={subsubsubsection}]

If you want to include every section depth, just remove the
\type{\setupcombinedlist[content][]} command.

\subsubsubsubsection[title={Subsubsubsubsection},reference={subsubsubsubsection}]

\subsection[title={Tables},reference={tables}]

\startplacetable[title={This is the table caption}]
\startxtable
\startxtablehead[head]
\startxrow
\startxcell[align=left] Right \stopxcell
\startxcell[align=right] Left \stopxcell
\startxcell Default \stopxcell
\startxcell[align=middle] Center \stopxcell
\stopxrow
\stopxtablehead
\startxtablebody[body]
\startxrow
\startxcell[align=left] 12 \stopxcell
\startxcell[align=right] 12 \stopxcell
\startxcell 12 \stopxcell
\startxcell[align=middle] 12 \stopxcell
\stopxrow
\startxrow
\startxcell[align=left] 123 \stopxcell
\startxcell[align=right] 123 \stopxcell
\startxcell 123 \stopxcell
\startxcell[align=middle] 123 \stopxcell
\stopxrow
\stopxtablebody
\startxtablefoot[foot]
\startxrow
\startxcell[align=left] 1 \stopxcell
\startxcell[align=right] 1 \stopxcell
\startxcell 1 \stopxcell
\startxcell[align=middle] 1 \stopxcell
\stopxrow
\stopxtablefoot
\stopxtable
\stopplacetable

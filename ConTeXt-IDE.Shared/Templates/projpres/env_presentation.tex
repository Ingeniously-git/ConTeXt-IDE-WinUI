\startenvironment[env_presentation]

% Basic Layout
\setupcolors[state=start]
\definepapersize[W1][width=192mm, height=108mm]
\mainlanguage[en] % Header names
\language[en] % Hyphenation
\enableregime[utf]
\setupalign[flushleft]

\startnotmode[handout]
	\setuppapersize[W1][W1]
	\setuplayout[topspace=0.2cm]
\stopnotmode

\startmode[handout] % Printer-friently layout
\definepapersize[handout][width=210mm,height=297mm][topspace=3cm]
	\setuplayout[handout][topspace=2cm, backspace=2cm, header=24pt]
	\setuppapersize[W1][A4] % This will create space for lecture notes below the actual page content
	\setuplayout[location={middle,top}]
	\setupbackgrounds[page][background=color, framecolor=gray, frame=on]
\stopmode

\setuplayout[
	backspace=0.4cm,
	width=fit,
	header=1cm,
	footer=0.6cm,
	cutspace=0cm,
	margindistance=0cm,
	margin=0cm,
	edgedistance=0cm,
	edge=0.4cm,
	headerdistance=0.4cm,
	footerdistance=0.4cm,
	bottom=0cm,
	top=0.2cm,
	topskip=0cm,
	topdistance=0cm,
	bottomdistance=0cm,
	bottomspace=0cm,
	height=fit,
	]

\setuppagenumbering[alternative=onesided, page=yes]

%\showgrid
%\showframe
%\showstruts
%\showmakeup
%\showsetups
%\showlayout

% PDF Output Adjustments
\setupinteraction[state=start, page=yes, menu=on, focus=fit, style=, click=yes, color=, contrastcolor=]
\placebookmarks[section,subsection,subject][section][force=yes]

% Accent color
\definecolor[accent][r=\ctxlua{context(45/255)},g=\ctxlua{context(137/255)},b=\ctxlua{context(204/255)}]
\definecolor[b][r=0.95,g=0.95,b=0.98]
\definecolor[f][r=0.8,g=0.8,b=0.8]

% This will create a screen-friendly version of the presentation when the mode screen is on and a printer-friendly version if it is off.
\startnotmode[print,handout] 
	%\setupinteractionmenu[state=start, maxheight=\measure{medium}]
	\setupinteractionscreen[option=max, width=fit, height=fit]
	\setupbackgrounds[page][background=color, backgroundcolor=b]
	\setupbackgrounds[footer][background=color, backgroundcolor=f]
	\defineoverlay[go-on][\overlaybutton{nextpage}]
	\setupbackgrounds[page][background=go-on]
\stopnotmode

% Typography
\setupindenting[no]
\setupbodyfont[sans, 11pt]
\definefontfeature[default][mode=node, kern=yes, bliga=yes, tlig=yes, ccmp=yes, language=dflt, protrusion=quality, expansion=quality]
\setuptolerance[verytolerant,stretch]

% Header/Footer
\setupfooter[style=\tf, location=high]
\setupheader[style=\tf, location=high]

\startsetups[prshead]
	\startframed[frame=off, align=flushleft, location=high]
		{\bfa\getmarking[sectionnumber]~\getmarking[section]}\\
		{\bf\getmarking[subsectionnumber]~\getmarking[subsection]}
	\stopframed
\stopsetups
\startsetups[prsheadfront]
	\startframed[frame=off, align=flushleft, location=high]
		{\bfa\getmarking[section]}\\
		{\bf\getmarking[subsection]}
	\stopframed
\stopsetups
\startsetups[prsheadback]
	\startframed[frame=off, align=flushleft, location=high]
		{\bfa\getmarking[subject]}\\
		{\bf\getmarking[subsubject]}
	\stopframed
\stopsetups
\startsetups[footerbar]
	\startframed[frame=off, align=low, location=bottom]
		\startnotmode[print,handout]
			%\interactionbar[alternative=a,width=.3\makeupwidth,step=small,symbol=yes]
		\doifelse{\lastpage}{0}{\definemeasure[percent][0pt]}{\definemeasure[percent][0.3\textwidth*\currentpage/\lastpage]}%
		\def\length{\the\textwidth*0.3}%
		\def\height{\the\footerheight}%
			\startMPcode{simplefun}
		beginfig(1)
		pair A, B, C, D;
		A:=(0,0); B:=(\length,0); C:=(\length,\height); D:=(0,\height);
		fill A--B--C--D--cycle withcolor .6 white;
		pair E, F, G, H;
		E:=(0,0); F:=(\measure{percent},0); G:=(\measure{percent},\height); H:=(0,\height);
		fill E--F--G--H--cycle withcolor .5 white;
		endfig;
		\stopMPcode \blank[-2mm]
		\stopnotmode
	\stopframed
\stopsetups
\startsetups[footertext]
	\startframed[frame=off, align=center, location=bottom]
		\getvariable{meta}{title} \| \getvariable{meta}{author}
	\stopframed
\stopsetups

\setupsubpagenumber[way=bysubsection, state=start, numberconversion=characters] % subpage counting is just for fun in this example, but might be usable for some komplex interactionbar visualization

% Section blocks layouting
\definestructureconversionset[frontpart:pagenumber][][Romannumerals]
\definestructureconversionset[bodypart:pagenumber][][numbers]
\definestructureconversionset[appendix:pagenumber][][Romannumerals]
\definestructureconversionset[backpart:pagenumber][][Romannumerals]
\setupsectionblock[frontpart][number=no]
\setupsectionblock[appendix][number=no]
\setupsectionblock[backpart][number=no]

\startsectionblockenvironment[frontpart]
%	\setnumber[userpage][0]
	\setupheadertexts[]
	\setupheadertexts[\directsetup{prsheadfront}][]
	\setupfootertexts[\directsetup{footertext}]
  \setupfootertexts[][\userpagenumber]
\stopsectionblockenvironment
\startsectionblockenvironment[bodypart]
	%\setupinterlinespace[distance=1ex, line=3ex]
	\setupheadertexts[]
	\setupheadertexts[\directsetup{prshead}][]
	\setupfootertexts[\directsetup{footertext}]
	\setupfootertexts[\directsetup{footerbar}][\userpagenumber\doifmode{screen}{\doifnotmode{print}{\subpagenumber}}]
	\savenumber[userpage]
	\resetnumber[userpage]
\stopsectionblockenvironment
\startsectionblockenvironment[appendix]
	\setupheadertexts[]
	\setupheadertexts[\directsetup{prsheadfront}][]
	\setupfootertexts[\directsetup{footertext}]
	\setupfootertexts[][\userpagenumber]
	\restorenumber[userpage]
	\incrementnumber[userpage] % This is to compenesate a bug
	\decrementnumber[userpage]
\stopsectionblockenvironment
\startsectionblockenvironment[backpart]
  \setupheadertexts[]
  \setupheadertexts[\directsetup{prsheadback}][]
	\setupfootertexts[\directsetup{footertext}]
  \setupfootertexts[][\userpagenumber]
\stopsectionblockenvironment

\definemakeup[title][doublesided=no, page=right, headerstate=empty, footerstate=empty,  pagestate=start, style=bigbodyfont, align=middle]

\definestartstop[newsection][before={\startstyle[tfa]\startframed[frame=off, width=max, height=max, location=bottom]}, after={\stopframed\stopstyle\page}]

% TOC style
\setuplist[part][width=0mm, distance=2mm, style=bold, aligntitle=no]
\setuplist[chapter][label=yes, width=fit, distance=2mm, style=bold]
\setuplist[section,subject][width=0mm, distance=2mm, margin=0mm, style=bold, alternative=b]
\setuplist[subsection,subsubject][width=0mm, distance=2mm, margin=5mm, alternative=c]
%\setuplist[subsubsection,subsubsubject][width=0mm, distance=2mm, margin=10mm, alternative=c]

\setupcombinedlist[content][list={section,subsection}]

\let\placetoc\placecontent
\define\placecontent{
\setupbackgrounds[header][text][frame=off, bottomframe=on]
\setmarking[section]{Content} % \setmarking[section]{} will set the header text without calling \section{}
\bookmark[section]{Content}
\placetoc
}

\defineparagraphs[column][n=2,]
\setupparagraphs[column][1][width=.5\textwidth,align=flushleft]

% Heading setup
\setuphead[chapter,title][style={\bfd}]
\setuphead[section][style={\bfc}, before=, after=, page=yes, placehead=empty]
\setuphead[subsection][style={\bfc}, before=, after=, page=yes, placehead=empty]
\setuphead[subject][style={\bfc}, before=, after=, page=yes, placehead=empty]
\setuphead[subsubject][style={\bfc}, before=, aftersection={\page}, page=no, placehead=empty]
%\setuphead[subsection,subsubject][style={\bfb}, before={}, after={}]
%\setuphead[subsubsection,subsubsubject][style={\bfa}, before={\bigskip}, after={}]

% Float captions
\setupcaptions[table][location=top]
\setupcaptions[figure][location=bottom]
\setupcaptions[headstyle=\bf]

% Referencing automation: \ref[Type:ReferenceName], somewhat equivalent to \autoref in LaTeX
\definereferenceformat[refdef][style=\tf, text=definition ]
\definereferenceformat[refthm][style=\tf, text=theorem ]
\definereferenceformat[refrmk][style=\tf, text=remark ]
\definereferenceformat[refsec][style=\tf, text=section ]
\definereferenceformat[refeq][style=\tf, text=equation ]
\definereferenceformat[reffig][style=\tf, text=figure ]
\definereferenceformat[reftab][style=\tf, text=table ]
\def\ref[#1:#2]{\csname ref#1\endcsname[#1:#2]}

% Bibliography
\usebtxdataset[default][literature.lua]
\setupbtx[dataset=default]
\usebtxdefinitions[aps]
\stopenvironment

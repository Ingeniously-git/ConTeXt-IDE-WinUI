\startproduct[prd_presentation]
\usemodule[present-steps] % modified stepping module

\startnotmode[print]
	\environment[env_presentation]
\stopnotmode

\startmode[print]
	\environment[env_script]
\stopmode

\setvariables[meta][title={Title}, subtitle={Subtitle}, author={Author}, date={\currentdate}]

\setupinteraction[title=\getvariable{meta}{title}, subtitle=\getvariable{meta}{subtitle}, author=\getvariable{meta}{author}, date=\getvariable{meta}{date},]

\starttext

\startfrontmatter
\starttitlemakeup
	{\bfd \getvariable{meta}{title}}
	\blank[3*medium]
	{\tfb \getvariable{meta}{subtitle}}
	\blank[3*medium]
	{\tfa \getvariable{meta}{author}}
	\blank[2*medium]
	{\tfa \getvariable{meta}{date}}
	\blank[3*medium]
	This \ConTeXt\ project is specifically designed for teachers/professors/tutors who want to generate both a presentation (with \& without stepping) and a handout/script compiled from the same code. Try out any permutation of the modes \quote{screen} and \quote{print} to see how the different outputs are produced.
\stoptitlemakeup

\placecontent

\blank[medium]
{\tfxx (The ToC depth is set in env_presentation.tex (env_script.tex respectively) with \type{\setupcombinedlist[content][]}. 

For example, \type{\setupcombinedlist[content][list={section,subsection}]} shows all sections and subsections. 

Replace it with \type{\setupcombinedlist[content][list={section}]} to show only the sections. 

Replace it with \type{\setupcombinedlist[content][list={section,subsection,subject,subsubject}]} to also show the backup slides in the ToC)}

\stopfrontmatter

\startbodymatter
\startsection[title=Stepping is cool]
\startsubsection[title=Basic stepping]
\startsteps
	\startstep
		{\bfa ! To enable/disable stepping, please enable/disable the mode \quote{screen} !}

		{\bfa ! To print the handout, use the mode \quote{print} !}

		STEP ONE. 

		Step.Substep :  \currentstep.\currentsubstep 

		Page.SubPage :  \userpagenumber.\subpagenumber

	\stopstep

	\startstep[n=4]
	\blank
		STEP Two. Note that the automatic increment of \type{\pagenumber} is stopped during stepping (see \type{\setuppagenumber[state=stop]} in the module). Instead of incrementing \type{\pagenumber}, a \type{\subpagenumber} (here shown in characters) is used, which gets resetted by subsection.

		Step.Substep :  \currentstep.\currentsubstep 
		\startitemize[horizontal,two,packed,broad]
			\startsubstep \startitem 1 \stopitem \stopsubstep
			\startsubstep \startitem 2 \stopitem \stopsubstep
			\startsubstep \startitem 3 \stopitem \stopsubstep
			\startsubstep \startitem 4 \stopitem \stopsubstep
		\stopitemize		
	\stopstep

	\startstep
		STEP THREE
	\stopstep
\stopsteps
\stopsubsection

\startsubsection[title=Columns]
\startsteps
\startstep[n=4]
hsize=\the\hsize
	\startcolumn
		\startsubstep 
			first column \blank		
			\starttabulate[|r|c|l|][unit=1ex]
			\NC columnwidth \NC= \NC\the\columnwidth \NC \NR
			
			\NC textwidth\NC =\NC \the\textwidth \NC \NR
			
			\NC makeupwidth\NC =\NC \the\makeupwidth \NC \NR
			
			\NC hsize\NC =\NC \the\hsize\NC \NR
			\stoptabulate
		\stopsubstep
		\startsubstep	
			\blank It seems that we need \type{\hsize} if we want to stretch floats to the column width.
		\stopsubstep
		\startsubstep 
			\column second column with a stretched picture (\ref[fig:cow])
		\stopsubstep
		\startsubstep
			\startplacefigure[title={Some figure}, reference=fig:cow]
				\startframed[width=\hsize, height=.4\hsize] % or \externalfigure[]
					< some figure >
				\stopframed
			\stopplacefigure
		\stopsubstep
	\stopcolumn
\stopstep
\stopsteps
\stopsubsection
\page[no]
\startsubsection[title=Tables]
\startsteps
	\startstep
		Tables are also working
	\stopstep
	\startstep[n=4]
		\startplacetable[location=here, reference=tab:exampletable, title={Example table}]
			\bTABLE[split=repeat, option=stretch, frame=off]
			\setupTABLE[r][each][style=\tfx\it, align=center]
			\setupTABLE[r][first][style=bold, align=center, bottomframe=on]
			\startsubstep
			\bTABLEhead
				\bTR \bTH head1 \eTH \bTH head2 \eTH \eTR
			\eTABLEhead
			\stopsubstep
			\bTABLEbody
			\startsubstep
			\bTR \bTD One \eTD \bTD Two \eTD \eTR
			\stopsubstep
			\bTR
			\startsubstep
				\bTD Three \eTD
			\stopsubstep
			\startsubstep
				\bTD Four \eTD
			\stopsubstep
			\eTR
			\eTABLEbody
			\eTABLE
		\stopplacetable
		\setupcaptions[table][state=stop] % Quick hack do disable the table counter
	\stopstep
	\startstep 
		Some more text
	\stopstep
\stopsteps
\page
A page with no steps, but in the same \quote{\getmarking[subsectionnumber]~\getmarking[subsection]} subsection.
\stopsubsection
\stopsection

\startsection[title={Another section}]
\startnewsection
In this section you will learn xyz!
\stopnewsection
\startsubsection[title={Another subsection}]
	Another page with no steps. \cite[hagen]
\stopsubsection
\startsubsection[title={Yet another subsection}]
Another page with no steps but with \color[accent]{keywords} that are \color[accent]{highlighted} in an \color[accent]{accent color} defined in env_presentation.
\stopsubsection
\stopsection

\stopbodymatter

\startappendices
\startsection[title={Appendix}]
\startsubsection[title={Formula signs}]
\startcolumns[n=2]
\bTABLE[split=yes,option=stretch, frame=off]
\setupTABLE[r][first][bottomframe=on]
	\bTABLEhead
		\bTR	\bTH  Symbol \eTH	\bTH  Unit \eTH	\bTH Description \eTH	\eTR
	\eTABLEhead
	\bTABLEbody
		\bTR \bTD $\rho_S$ \eTD \bTD \unit{kilogram per cubic meter} \eTD	\bTD Solid phase density \eTD \eTR
		\bTR \bTD $\dot{V}_{in}$ \eTD \bTD \unit{cubic meter per hour} \eTD \bTD Ingoing volume flow \eTD \eTR
		\bTR \bTD $\rho_S$ \eTD \bTD \unit{kilogram per cubic meter} \eTD	\bTD Solid phase density \eTD \eTR
		\bTR \bTD $\dot{V}_{in}$ \eTD \bTD \unit{cubic meter per hour} \eTD \bTD Ingoing volume flow \eTD \eTR
		\bTR \bTD $\rho_S$ \eTD \bTD \unit{kilogram per cubic meter} \eTD	\bTD Solid phase density \eTD \eTR
		\bTR \bTD $\dot{V}_{in}$ \eTD \bTD \unit{cubic meter per hour} \eTD \bTD Ingoing volume flow \eTD \eTR
		\bTR \bTD $\rho_S$ \eTD \bTD \unit{kilogram per cubic meter} \eTD	\bTD Solid phase density \eTD \eTR
		\bTR \bTD $\dot{V}_{in}$ \eTD \bTD \unit{cubic meter per hour} \eTD \bTD Ingoing volume flow \eTD \eTR
		\bTR \bTD $\rho_S$ \eTD \bTD \unit{kilogram per cubic meter} \eTD	\bTD Solid phase density \eTD \eTR
		\bTR \bTD $\dot{V}_{in}$ \eTD \bTD \unit{cubic meter per hour} \eTD \bTD Ingoing volume flow \eTD \eTR
		\bTR \bTD $\rho_S$ \eTD \bTD \unit{kilogram per cubic meter} \eTD	\bTD Solid phase density \eTD \eTR
		\bTR \bTD $\dot{V}_{in}$ \eTD \bTD \unit{cubic meter per hour} \eTD \bTD Ingoing volume flow \eTD \eTR
		\bTR \bTD $\rho_S$ \eTD \bTD \unit{kilogram per cubic meter} \eTD	\bTD Solid phase density \eTD \eTR
		\bTR \bTD $\dot{V}_{in}$ \eTD \bTD \unit{cubic meter per hour} \eTD \bTD Ingoing volume flow \eTD \eTR
		\bTR \bTD $\rho_S$ \eTD \bTD \unit{kilogram per cubic meter} \eTD	\bTD Solid phase density \eTD \eTR
		\bTR \bTD $\dot{V}_{in}$ \eTD \bTD \unit{cubic meter per hour} \eTD \bTD Ingoing volume flow \eTD \eTR
	\eTABLEbody
\eTABLE
\stopcolumns
\stopsubsection
\startsubsection[title={Literature}]
	\placelistofpublications
\stopsubsection
\stopsection
\stopappendices

\startmode[screen]
\startbackmatter 
\startsubject[title={Backup slides}]
\startsubsubject[title={Additional presentable stuff}]
	Only shown in screen mode.
\stopsubsubject
\startsubsubject[title={More presentable stuff}]
	Only shown in screen mode..
\stopsubsubject
\stopsubject
\startsubject[title={More backup slides}]
\startsubsubject[title={Additional presentable stuff}]
	Only shown in screen mode...
\stopsubsubject
\startsubsubject[title={More presentable stuff}]
	Only shown in screen mode....
\stopsubsubject
\stopsubject

\startnotmode[print]
\page
\setupbackgrounds[footer][background=, backgroundcolor=]
\setupbackgrounds[bottom][background=, backgroundcolor=]
\starttitlemakeup
	\bfd Thanks for your attention!
\stoptitlemakeup
\stopnotmode

\stopbackmatter
\stopmode

\stoptext

\stopproduct

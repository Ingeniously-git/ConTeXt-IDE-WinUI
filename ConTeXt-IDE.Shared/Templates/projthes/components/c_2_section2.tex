\startcomponent[c_2_section2]
\environment[env_thesis]

\startsection[title=Math]
The formula notation is the same as in \LaTeX, but alignment and label assignment is done \quote{The \ConTeXt\ way}.

This is a normal formula with the usual \ConTeXt-like referencing scheme: \ref[eq:F]
\startplaceformula[reference=eq:F]
	\startformula
		F = m \cdot a
	\stopformula
\stopplaceformula

These are aligned equations. Note that if we use aligned math, the label of each formula needs to be set as an argument of \type{\NR} (\quote{New Row}). This is a reference to \ref[eq:dcdt]. 

\startplaceformula
	\startsubformulas
		\startformula
			\startalign
				\NC \frac{dX}{dt} \NC = k \cdot (X_{eq} - X) \NR[eq:dXdt]
				\NC \frac{dc}{dt} \NC = - \frac{v_0}{\varepsilon} \cdot \frac{dc}{dx} + \frac{1}{\varepsilon} \cdot \frac{dX}{dt} \NR[eq:dcdt]
			\stopalign
		\stopformula
	\stopsubformulas
\stopplaceformula

We can also align seperately numbered formulas. Note that you need to set a label on every equation to be numbered:
\startplaceformula
	\startformula
		\startalign
			\NC \frac{dX}{dt} \NC = k \cdot (X_{eq} - X) \NR[eq:dcdt2]
			\NC \frac{dc}{dt} \NC = - \frac{v_0}{\varepsilon} \cdot \frac{dc}{dx} + \frac{1}{\varepsilon} \cdot \frac{dX}{dt} \NR[eq:dXdt2]
		\stopalign
	\stopformula
\stopplaceformula

You could also number and reference the whole group (\ref[eq:group]) instead of individual formulas:
\startplaceformula[reference=eq:group]
	\startformula
		\startalign
			\NC \frac{dX}{dt} \NC = k \cdot (X_{eq} - X) \NR
			\NC \frac{dc}{dt} \NC = - \frac{v_0}{\varepsilon} \cdot \frac{dc}{dx} + \frac{1}{\varepsilon} \cdot \frac{dX}{dt} \NR
		\stopalign
	\stopformula
\stopplaceformula

\startsubsection[title={Units}]
	Do you need to pretty-print units? No problem: $v_0=\unit{0.5 meter per second}$. Please always use \type{\unit{}} to print value-unit pairs as the neccessary space is automatically filled inbetween. And more importantly, the unit letters are printed upright to be easily destinguished from the italicized formula symbols.
\stopsubsection

\startsubsection[title={Theorems}]
For the definition of the following enumerations see the env_thesis.tex.
\startdefinition
	This is a definition.
\stopdefinition

\starttheorem
	This is a theorem.
\stoptheorem

\startremark
	This is a remark.
\stopremark
\stopsubsection

\startsubsection[title={TikZ Graphics}]
This is a controvertial topic because \ConTeXt\ users usually use \MetaPost\ to draw graphics. But in physics, engineering and related disciplines we need the power of tikz/pgfplots if we want to present data directly compiled in \TeX. But be aware that compile time increases dramatically! This is why I defined a draft mode in env_thesis.tex where the tikz pictures are not getting rendered.

\pgfplotsset{ % here you can define the appearance of your plots globally
	major tick length=2mm,
	minor tick length=1mm,
	minor tick num=1,
	every tick/.style={black,thick,},
	minor tick/.style={gray,semithick,},
}

\startplacefigure[title={Crazy stuff you can do with pgfplots}, location=here, reference=fig:combinations]
	\startfloatcombination[2*2]
		\startplacesubfigure[title={2D, axes build a frame}]
			\startframed[frame=off,offset=none]
				\starttikzpicture
					\startaxis[xmin=-5, xmax=15, ymin=-5, ymax=5, domain=-5:15, xtick distance={5}, ytick distance={5}, width=.5\columnwidth, xlabel=x, ylabel=y, legend pos=south east]
						\addplot[mark=+, blue,] {cos(deg(x))};
						\addplot[mark=*, red, dashed] {sin(deg(x))};
						\addplot[mark=square]{5*(1-exp(-0.5*x))};
    				\legend{$\sin(x)$,$\cos(x)$,$x^2$}
					\stopaxis
				\stoptikzpicture
			\stopframed
		\stopplacesubfigure
		\startplacesubfigure[title={2D, axes go through the origin}]
			\startframed[frame=off,offset=none]
				\starttikzpicture
					\startaxis[xmin=-5, xmax=15, ymin=-5, ymax=5, domain=-5:15, xtick distance={5}, ytick distance={5}, width=.5\columnwidth, xlabel=x, ylabel=y, legend pos=south east, axis lines=middle,  axis line style = {-latex}, grid=both, minor grid style={draw=gray!50},]
						\addplot[mark=+, blue,] {cos(deg(x))};
						\addplot[mark=*, red, dashed] {sin(deg(x))};
						\addplot[mark=square]{5*(1-exp(-0.5*x))};
    				\legend{$\sin(x)$,$\cos(x)$,$x^2$}
					\stopaxis
				\stoptikzpicture
			\stopframed
		\stopplacesubfigure
		\startplacesubfigure[title={3D, but rotated to show the 2D heatmap}]
			\startframed[frame=off,offset=none]
				\starttikzpicture
					\startaxis[view={0}{90}, domain=-180:180, samples=30, width=.5\columnwidth, colormap/viridis, xlabel=x, ylabel=y, zlabel=z]
						\addplot3 [surf, shader=faceted] {sin(x) * sin(y)};
					\stopaxis
				\stoptikzpicture
			\stopframed
		\stopplacesubfigure
		\startplacesubfigure[title={3D}]
			\startframed[frame=off,offset=none]
				\starttikzpicture
					\startaxis[view={60}{40}, domain=-180:180, samples=30, width=.5\columnwidth, colormap/bluered, xlabel={$x [\unit{meter}]$}, ylabel={$t [\unit{second}]$}, zlabel={$c [\unit{gram per cubic meter}]$}]
						\addplot3 [surf, shader=flat] {sin(x) * sin(y)};
					\stopaxis
				\stoptikzpicture
			\stopframed
		\stopplacesubfigure
	\stopfloatcombination
\stopplacefigure

\useURL[pgfplotsgallery][http://pgfplots.sourceforge.net/gallery.html]
\useURL[pgfplotsmanual][http://mirrors.ctan.org/graphics/pgf/contrib/pgfplots/doc/pgfplots.pdf]
For a huge list of example plots see \from[pgfplotsgallery].

For the full reference, see \from[pgfplotsmanual].

\stopsubsection
\stopsection
\stopcomponent